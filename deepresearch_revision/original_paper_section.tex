\section{客観的生体反応分析: フェーズ別統計および指標分析}
収集された生体信号 (EDA, ECG, EMG) について, 各フェーズごとの統計的変化と特徴的な反応パターンを分析する.

\subsection{皮膚電気活動 (EDA) 分析結果}
皮膚電気活動は交感神経系の活性度を表す核心的な指標であり,本実験ではSCLの平均値を中心に分析した.

\begin{table}[t]
\centering
\caption{全参加者のフェーズ別EDA (SCL) 統計値 (Mean $\pm$ SD)}
\label{tab:eda_scl_stats}
\begin{tabular}{lcccc}
\hline
フェーズ (Phase) & 平均 SCL ($\mu$S) & 標準偏差 (SD) & 最小値 (Min) & 最大値 (Max) \\
\hline
Phase 0 & 22.20 & 6.22 & 9.05 & 28.02 \\
Phase 1  & 20.90 & 6.51 & 7.55 & 28.09 \\
Phase 2  & 26.59 & 2.99 & 19.42 & 28.20 \\
Phase 3  & 26.67 & 1.47 & 23.94 & 27.82 \\
\hline
\end{tabular}
\end{table}

データ分析の結果,Phase 0とPhase 1の間では有意なSCLの上昇は観察されなかったが,暗い屋敷を探索し始めるPhase 2に進入すると,SCL平均値が26.59 $\mu$Sへと急激に上昇した.SCL平均値の上昇は,視覚的制限と環境音が参加者の警戒心を高め,交感神経系を持続的に覚醒させたことを意味する.Phase 3でも26.67 $\mu$Sの高い水準が維持され,標準偏差が1.47 $\mu$Sと非常に低く現れた点は,すべての参加者が共通して高い覚醒状態に到達したことを示している.特にSCRピーク数については,Phase 1で平均7.88回であったものが,探索と追跡が含まれるシナリオにおいて持続的に発生し,脅威刺激に対する反応性を証明した.

% ============================================
\subsection{心拍変動 (HRV) および心電図 (ECG) 分析結果}
心電図データを基に算出された心拍数と心拍変動指標は,参加者の自律神経系バランスとストレスレベルを表す. 表\ref{tab:ecg_hrv_stats}に全参加者のフェーズ別ECG/HRV主要統計値を示す.

\begin{table}[t]
\centering
\caption{全参加者のフェーズ別ECG/HRV主要統計値 (Mean $\pm$ SD)}
\label{tab:ecg_hrv_stats}
\begin{tabular}{lcccc}
\hline
特徴量 (Feature) & Phase 0 & Phase 1 & Phase 2 & Phase 3 \\
\hline
平均心拍数 (HR, bpm) & 73.02 $\pm$ 11.65 & 74.15 $\pm$ 9.54 & 76.52 $\pm$ 8.11 & 72.16 $\pm$ 20.23 \\
RMSSD (ms) & 143.34 $\pm$ 284.51 & 51.42 $\pm$ 32.71 & 67.04 $\pm$ 53.03 & 742.37 $\pm$ 1775.91 \\
SDNN (ms) & 126.01 $\pm$ 193.24 & 54.86 $\pm$ 32.56 & 68.30 $\pm$ 35.40 & 459.45 $\pm$ 1051.43 \\
LF/HF比 & 1.72 $\pm$ 1.10 & 2.37 $\pm$ 1.95 & 1.87 $\pm$ 1.26 & 1.58 $\pm$ 2.50 \\
\hline
\end{tabular}
\end{table}

生理学的指標は個人間で大きな変動を示すため,各参加者のPhase 0 (安静状態) における測定値をベースラインとして使用し,以降のフェーズにおける値をベースラインからの変化率として算出した(表\ref{tab:ecg_hrv_normalized}).具体的には, 以下の式に示すように, 各フェーズの測定値をベースライン値と比較し, パーセント変化率に変換した.


$$ \Delta X_{\mathrm{norm}} (\%) = \frac{X_{\mathrm{Phase}_n} - X_{\mathrm{Phase}_0}}{X_{\mathrm{Phase}_0}} \times 100 $$

\begin{table}[t]
\centering
\caption{ベースライン正規化後のフェーズ別ECG/HRV変化率 (\%, Mean $\pm$ SD)}
\label{tab:ecg_hrv_normalized}
\begin{tabular}{lccc}
\hline
特徴量 & Phase 1 & Phase 2 & Phase 3 \\
\hline
心拍数変化率 & +1.5 $\pm$ 13.1 & +4.8 $\pm$ 11.1 & -1.2 $\pm$ 27.7 \\
RMSSD変化率 & -64.1 $\pm$ 22.8 & -53.2 $\pm$ 37.0 & +417.9 $\pm$ 1239.0 \\
SDNN変化率 & -56.5 $\pm$ 25.8 & -45.8 $\pm$ 28.1 & +264.6 $\pm$ 834.4 \\
LF/HF変化率 & +37.3 $\pm$ 113.1 & +8.8 $\pm$ 73.1 & -8.1 $\pm$ 145.0 \\
\hline
\end{tabular}
\end{table}

正規化された結果から,以下の傾向が確認された. 心拍数はPhase 2で平均+4.8\%の上昇を示し,探索過程における持続的な緊張状態を反映した. Phase 3では平均-1.2\%と微減したが,標準偏差が27.7\%と顕著に増大しており,「すくみ反応 (Freezing Response)」による心拍数低下とパニック反応による心拍数急増が参加者間で混在した結果であると解釈される.

時間領域HRV指標であるRMSSDとSDNNは,Phase 1においてそれぞれ-64.1\%,-56.5\%と大幅に減少した. これは認知的負荷が副交感神経活動を抑制し,ストレス反応を誘導したことを定量的に示している. Phase 3における異常な上昇 (+417.9\%, +264.6\%) は,一部参加者の激しい不整脈様反応や身体動作によるアーティファクトが反映された結果と推定され,今後の分析において精密なノイズ除去の必要性を示唆している.

周波数領域のLF/HF比はPhase 1で+37.3\%と最も顕著な上昇を示し,Go/No-Go課題における高度な集中と交感神経系の活性化を意味する. Phase 2以降は漸減し,Phase 3では-8.1\%とベースラインをやや下回った.



\subsection{筋電図 (EMG) 分析結果}
右腕の筋肉に装着されたEMGセンサは,参加者の身体的緊張度と驚き反応に伴う筋肉収縮を測定した.

\begin{table}[t]
\centering
\caption{全参加者のフェーズ別EMG主要統計値 (Mean $\pm$ SD)}
\label{tab:emg_stats}
\begin{tabular}{lcccc}
\hline
特徴量 (Feature) & Phase 0 & Phase 1 & Phase 2 & Phase 3 \\
\hline
平均電圧 (Mean, $\mu$V) & 14.90 $\pm$ 7.17 & 15.95 $\pm$ 8.36 & 19.83 $\pm$ 7.94 & 27.69 $\pm$ 17.36 \\
RMS電圧 (RMS, $\mu$V) & 16.87 $\pm$ 6.63 & 18.23 $\pm$ 6.61 & 25.40 $\pm$ 8.93 & 36.01 $\pm$ 30.05 \\
最大電圧 (Max, $\mu$V) & 81.86 $\pm$ 60.27 & 104.03 $\pm$ 66.30 & 228.27 $\pm$ 190.88 & 223.11 $\pm$ 302.27 \\
\hline
\end{tabular}
\end{table}

EMG分析の結果,フェーズが進行するにつれて筋肉の緊張度が明確に上昇する様相が観察された.特にPhase 3におけるRMS電圧は36.01 $\mu$Vであり,Baseline (16.87 $\mu$V) と比較して2倍以上に増加し,最大電圧は平均223.11 $\mu$Vになった.EMGの増加は,ゾンビの追跡という直接的な脅威の前で,被験者が逃避あるいは防御のために身体筋肉を強く収縮させたことを裏付ける証拠である.また,標準偏差が30.05 $\mu$Vと非常に大きく現れた点は,被験者の傾向によって身体的硬直反応の程度が極端に分かれたことを示唆している.

\section{個人差分析および特異反応者の識別:主観・客観連携の深層事例}
全体統計分析を超え,主観的報告と生体反応の間の相関関係を明確に把握するため,主要な参加者3名に対する深層事例分析を実施した.

\subsection{特異な反応を示す参加者 (Subject E2)}
Subject E2は, アンケートにおいて追跡時の恐怖度7点(満点)と実験終了後の解放感7点(満点)を記録し, 実験刺激に対して強烈な心理的経験をしたことを報告した. 

補正後の生体信号データにおいても, 心拍数 (HR) は $100.4 \text{ bpm}$ と高い値を記録して強力な交感神経系の活性化を示し, 筋電図 (EMG) のRMS値は $57.01 \mu\text{V}$ と全被験者平均 ($36.0 \mu\text{V}$) を1.5倍以上上回った. また, 心拍変動 (RMSSD) は $154.4 \text{ ms}$ であった.

これらの結果から, Subject E2は高い心拍数とともに実験参加者中で最も高いレベルの筋緊張を示したといえる. これは恐怖刺激に対して身体が凍りつくのではなく, 即座の逃避行動のために全身の筋肉を緊張させ心拍出量を増加させる\textbf{「能動的逃避反応 (Active Flight Response)」}が発現したものと解釈される. 高い主観的恐怖度と一致するこの生体信号パターンは, VRコンテンツがユーザに実在感のある脅威を伝達することに成功したことを示す最も強力な証拠である.

\subsection{認知失敗および低反応者 (Subject D)}
参加者Dは,主観的に追跡を認知できなかったと報告しており,追跡を認知できなかったことは生体指標の停滞につながった.
\begin{itemize}
\item \textbf{SCL減少}: Phase 2 (28.20 $\mu$S) からPhase 3 (23.94 $\mu$S) へと覚醒数値が下落した.
\item \textbf{安定的HR}: 心拍数もまた77.77 bpmから78.65 bpmへとほとんど変化がなかった.
\end{itemize}
以上の結果は「実験課題が無事に終了した」と判断し,急激な心理的弛緩 (Early Relief) 状態に進入したことを示唆する.このようなベースライン以下への生体指標の下落は,感情推定フレームワークが単に恐怖の有無だけでなく,ユーザが現在の状況を「安全」あるいは「終了」と認知しているか否かを判別する有用な指標となり得ることを示している.

\subsection{無意識的緊張維持者 (Subject F)}
参加者Fは事後アンケートにおいて追跡を認知できなかったと回答したが,彼の身体は異なる反応を示していた.
\begin{itemize}
\item \textbf{高い覚醒水準}: Phase 3のSCL値は25.32 $\mu$Sであり, 全参加者のPhase 3平均である26.67 $\mu$Sに近接した数値である. 心拍数は90.39 bpmで実験全体の中で最高値を記録し, EMG RMSもまた35.17 $\mu$Vで平均以上の高い身体的緊張度を示した.
\item \textbf{SCRピークの存在}: 特にPhase 2において16回のSCRピークが発生したが, SCRピークの頻発は,参加者Fがアイテムを収集する過程ですでに相当な心理的圧迫を感じていたことを意味する.
\end{itemize}
「追跡を認知できなかった」という回答は,実際の追跡状況における視覚的認知失敗を意味するだけであり,彼が置かれていた環境全体に対する緊張感は解消されていなかったことを示唆する.
